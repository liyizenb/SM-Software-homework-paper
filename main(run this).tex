\documentclass[12pt]{HomeWork}

%%%%%% ========== 身份信息填写 ========== %%%%%%
\title{统计软件及应用作业}                   %题目
% \title{数学软件综合训练\bigroman{2}作业}   %题目(二选一)
\newcommand{\Class}{数学2xx}               %班级
\newcommand{\Lession}{统计软件01}          %序号
\newcommand{\Num}{2023123456789}          %学号
\author{李奕贝}                            %姓名
%%%%%%%%%%%%%%%%%%%%%%%%%%%%%%%%%%%%%%%%%%%%%%%

%%%%%% ========== 页眉设置 ========== %%%%%%
\fancyhead[C]{\thetitle{}}
\fancyhead[L]{\Class \theauthor{}}
\fancyhead[R]{\today}

\begin{document}
\thispagestyle{empty}

%%%%%% ========== 首页标题 ========== %%%%%%
\begin{center}
    \zihao{3}\heiti \thetitle{}
\end{center}

%%%%%% ========== 基本信息表格 ========== %%%%%%
\noindent
\begin{tabular}{@{}p{0.4\textwidth} p{0.4\textwidth}@{}}
    \quad 序号:\Lession & \centering 日期:\today \tabularnewline
\end{tabular}

\vspace{1em}
\noindent
\begin{tabular}{|>{\centering\arraybackslash}p{1.5cm}|
                  >{\centering\arraybackslash}p{2.6cm}|
                  >{\centering\arraybackslash}p{1.5cm}|
                  >{\centering\arraybackslash}p{2.6cm}|
                  >{\centering\arraybackslash}p{1.5cm}|
                  >{\centering\arraybackslash}p{3.4cm}|}
    \hline
    班级 & \Class & 姓名 & \theauthor{} & 学号 & \Num \\
    \hline
    \begin{minipage}[t][1.5cm][t]{\linewidth}
        \centering 章节\\ 名称
    \end{minipage} 
    &
    \multicolumn{5}{p{12.1cm}|}{%
        \begin{minipage}[t]{\linewidth}
            第1章\quad SPSS 软件概述\\
            第2章\quad 统计数据的收集与预处理\\
            第3章\quad 描述性统计分析
            \vspace{0.3em}
        \end{minipage}
    } \\
    \hline
\end{tabular}

%%%%%% ========== 正文内容 ========== %%%%%%
\section*{题目与求解}

\subsectionWithoutSectionNum    %勿删,调整序号用
\subsection{使用帮助}
\begin{itemize}
    \item 本文档代码开源,存放于 Overleaf 和 GitHub 网站,项目网址如下:
    \begin{itemize}
        \item \href{https://cn.overleaf.com/read/gqyrhvbcdphk\#93ff90}{\url{https://cn.overleaf.com/read/gqyrhvbcdphk\#93ff90}}
        \item \href{https://github.com/liyizenb/SM-Software-homework-paper}{\url{https://github.com/liyizenb/SM-Software-homework-paper}}
    \end{itemize}
    \item 文档适配大多数 LaTeX 编译环境,未特别说明部分与标准 LaTeX 一致;
    \item 由于 SPSS 中 ``*'' 为注释符号,与乘号冲突,因此在文档末尾的代码展示环境中用 ``*-'' 作为注释前缀。
\end{itemize}

\subsection{代码插入}
\begin{lstlisting}[language=SPSS, caption={SPSS 代码示例}]
*- 以下是计算总分的 SPSS 语句。
COMPUTE grade = 10*(Q1 = "a" + Q2 = "c" + Q3 = "d" + Q4 = "b" 
    + Q5 = "b" + Q6 = "a" + Q7 = "b" + Q8 = "b" + Q9 = "c" + Q10 = "a").
EXECUTE.
\end{lstlisting}
\begin{lstlisting}[language=Matlab, caption={Matlab 代码示例}]
r = 2;
p = 0.5;
n = 12;
T = log(r) / (n * log(1 + 0.01 * p));
disp(T)
\end{lstlisting}
\begin{lstlisting}[language=R, caption={R语言 代码示例}]
# --- 1. 录入数据 ---
data <- read.table(header = TRUE, text = "
ID Gender Math Chinese English Physics
3001 男 80 76 81 90
...
3035 女 64 70 72 77
", encoding = "UTF-8")
\end{lstlisting}

\subsection{作业3}
$$\cdots$$
% ……可继续补充内容……



%%%%%% ========== 小结与扩展 ========== %%%%%%
\section*{小结}
(可以写一下作业过程中遇到的问题及解决方案,以及需要注意的事项)

\begin{note}
    这是笔记环境
\end{note}

\begin{prb}
    这是问题环境
\end{prb}

\begin{soln}
    这是解决环境
\end{soln}

\end{document}
